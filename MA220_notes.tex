\documentclass[a4paper, 12pt]{article}

\usepackage[utf8]{inputenc}
\usepackage[T1]{fontenc}
\usepackage{textcomp}
\usepackage{amsmath, amssymb}
\usepackage[left=25mm, right=25mm, top=20mm, bottom=20mm]{geometry}

\pdfsuppresswarningpagegroup=1
\title{MA220 Linear Algebra. Block 4 2022}
\author{Reuben Alter}
\date{}
\begin{document}
\maketitle
\newpage
\tableofcontents

\newpage

\begin{section}{What is Linear Algebra}
\begin{subsection}*{Topics}
\begin{enumerate}
		\item{What Linear Algebra is}
		\item{Elementary Row Operations (ERO's)}
		\item{Echelon form}
\end{enumerate}
\end{subsection}
\begin{subsection}{What is Linear Algebra?}
\textbf{The Basic Definition:} Linear Algebra is the study of 
solving linear equations.\\ These generally come in the form of
$a_1*x_1+a_2*x_2+...+a_n*x_n=b$\\ We are looking for intersections.
This can be done through substitution, general algebraic methods 
(ERO's) and matrix rotation. \\
\end{subsection}
\begin{subsection}{Elementary Row Operations (Three Main Types)}
\begin{enumerate}
	\item{\textbf{Replacement}: Replace one row by the 
		sum of itself and a scalar multiple of another.}
	\item{\textbf{Interchange}: Switch two rows.}
	\item{\textbf{Scaling}: Multiply all entries in a row by
		a non-zero constant}
\end{enumerate}
\end{subsection}
\begin{subsection}{Echelon Forms}
	\textbf{Echelon Form}: All non-zero rows are above rows
	of all zeroes.
	\begin{itemize}
	\item{All leading entries of a row is in a column to 
		the right of the row above it}
	\item{All entries are in a column below a leading entry
		are zeroes}
\end{itemize}
\textbf{Reduced Echelon Form}: A unique solution!
\begin{itemize}
\item{The leading entry in each non-zero row is a 1}
\item{Each leading 1 is the only non-zero entry in the column}
\end{itemize}
\end{subsection}

\begin{subsection}{Miscellaneous Notes from Day 1}
	\begin{itemize}
		\item{\textbf{An Intersection of Lines}:
			unique solution}
		\item{\textbf{Equivalent Lines}:
			infinite solutions}
		\item{\textbf{Parallel lines}: no solution}
		\item{\textbf{Consistent}: $\geq 1$ solutions}
		\item{\textbf{Inconsistent}: 0 Solutions}
	\end{itemize}
\end{subsection}

\end{section}	

\begin{section}{Vector and Matrix Equations}

\begin{subsection}*{Theorems}
\textbf{Theorem 3}: If A is an m*n matrix with columns
$\textbf{a}_1, \textbf{a}_2,\dots,\textbf{a}_n$ and if 
$b \in \mathbb{R}^n$, then matrix equation $A\textbf{x}=\textbf{b}$
has the same solution as the vector equation
$x_1\textbf{a}_1+x_2\textbf{a}_2+\dots+x_n\textbf{a}_n=\textbf{b}$
and the system with the augmented matrix 
$\begin{bmatrix} 
\textbf{a}_1 & \textbf{a}_2 & \textbf{a}_n & \textbf{b} \end{bmatrix}$ 
\\ \\
\textbf{Theorem 4}: For an m*n matrix A, the following statements
are logically equivalent
\begin{enumerate}
	\item{For all $\textbf{b}\in\mathbb{R}^m$ 
	there exists a solution for $A\textbf{x}=\textbf{b}$}
	\item{Each $\textbf{b}\in\mathbb{R}^m$ is a linear combination
		of A columns}
	\item{The columns of A span $\mathbb{R}^m$}
	\item{A has a pivot in each row}
\end{enumerate}
\end{subsection}
\begin{subsection}*{Topics}
\begin{enumerate}
	\item{Defining vectors and the span of a collection of vectors}
	\item{Writing linear equations as a matrix * vector}
	\item{Recognizing $M*\textbf{v}$ as a linear combination
		of columns of a matrix}
	\item{Algebraic Properties of Vectors}
\end{enumerate}
\end{subsection}

\begin{subsection}{Span and vectors}
\textbf{Vectors}: A matrix with one column and m dimensions
where m is number of rows.
\begin{itemize}
	\item{$\textbf{u}=\langle1,2\rangle$ 
	defines $\mathbb{R}^2$}
\item{$\textbf{v}=\langle1,2,-5\rangle$ defines $\mathbb{R}^3$}
\end{itemize}
\textbf{Linear Combinations of Vectors}: a set where 
$\textbf{y}=c_1\textbf{v}_1+c_2\textbf{v}_2+\dots+c_n\textbf{v}_n$
where $(c_1,c_2,\dots,c_p)$ are scalars (weights)\\

\noindent\textbf{Span}: The set of all linear combinations Span
$\lbrace \textbf{v}_1 , \dots , \textbf{v}_p \rbrace$ where 
$\textbf{b}=c_1\textbf{v}_1+c_2\textbf{v}_2+\dots+
c_p\textbf{v}_p \in \mathbb{R}$
\begin{itemize}
	\item{e.g span of $\textbf{v} \in \mathbb{R}^2$ is a line}
	\item{e.g span of $\textbf{v}+\textbf{u} 
	\in \mathbb{R}^3$ is a plane}
\end{itemize}
By the definition of 
$A \textbf{x}=x_1 \textbf{a}_1+\dots + \textbf{x}_n \textbf{a}_n$ 
through Theorem 3,$A\textbf{x}=\textbf{b}$ has a solution iff 
\textbf{b} is the span of A columns.
\end{subsection}
\begin{subsection}{Algebraic Properties of Vectors}
	For all \textbf{u},\textbf{v},\textbf{w} $\in \mathbb{R}^n$
	where $c_1 , c_2$ are scalars, these are true:
\begin{enumerate}
	\item{$\textbf{v}+\textbf{u}=\textbf{u}+\textbf{v}
	\Rightarrow (\textbf{u}+\textbf{v})+\textbf{w}=
\textbf{u}+(\textbf{v}+\textbf{w})$}
\item{$\textbf{u}+\textbf{0}=\textbf{u}$}
\item{$c(\textbf{u}+\textbf{v})=c\textbf{u}+c\textbf{v}$}
\item{$(c+d)\textbf{u}=c\textbf{u}+d\textbf{u}$}
\item{There is no $\textbf{u}*\textbf{v}$}
\end{enumerate}
\end{subsection}
\end{section}

\newpage
\begin{section}{Homogenous Systems and Linear Independence}

\begin{subsection}*{Theorems}
\textbf{Theorem 6}: Suppose the equation $A\textbf{x}=\textbf{b}$ is
consistent for some \textbf{b}, then the solution set of all the 
vectors of the form is $\textbf{w}=\textbf{p}+\textbf{v}_n$, where
$\textbf{v}_n$ is any solution of the homogenous 
system $A\textbf{x}=\textbf{0}$
\\
\\ \textbf{Theorem 7}: A set of 2+ vectors $S:\lbrace\textbf{v}_1,\dots
,\textbf{v}_n\rbrace$ is linearly dependent (LD) iff $\geq 1$ vector 
in S is a linear combination of the others.\\
\\
\textbf{Theorem 8}:if a set $\in\mathbb{R}^n$ contains more vectors 
than dimensions, then the set is LD.\\
(This is because there will be a free variable and thus not a single
unique solution)
\\
\\ \textbf{Theorem 9}: If a set $S:\lbrace\textbf{v}_1,/dots
,\textbf{v}_n\rbrace \in\mathbb{R}^n$ contains a \textbf{0} vector, the
set is LD
\end{subsection}
\begin{subsection}*{Topics}
	\begin{enumerate}
	\item{Homogenous Systems}
	\item{Parametric Form}
	\item{Linear Independence and Dependence}
	\end{enumerate}
\end{subsection}
\begin{subsection}{Homogenous Systems}
When $A\textbf{x}=\textbf{0}$, the trivial solution is
$\textbf{x}=\textbf{0}$. The non-trivial form of this is when
\textbf{x} is equal to something other then \textbf{0}.\\
The solution set to a homogenous system is the span of some set of 
vectors $S:\lbrace\textbf{v}_1,\textbf{v}_n\rbrace$
\begin{itemize}
\item{The system $A\textbf{x}=\textbf{0}$
	will have a non-trivial solution if there is a free
	variable within the coefficient matrix}
\end{itemize}

\end{subsection}

\begin{subsection}{Parametric Form}
\begin{enumerate}
	\item{Row reduce augmented matrix to reduced echelon form}
	\item{Express base variables in terms of free variables in 
		equation system}
	\item{Write solution vector \textbf{x}whose entries
		depend on free variables}
	\item{decompose \textbf{x} into linear combination w/ free 
		variable as the parameter for one of the vectors}
\end{enumerate}
\begin{equation*}{Example:}
\begin{bmatrix} 
	1 & 0 & 2 & 2 \\
	0 & 1 & 3 & 4 \\
	0 & 0 & 0 & 0
\end{bmatrix} 
\Rightarrow
x_3\begin{bmatrix} -2 \\ -3 \\ 1 \end{bmatrix} +
\begin{bmatrix} 2 \\ 4 \\ 0 \end{bmatrix} 
\end{equation*}
\end{subsection}

\begin{subsection}{Linear Independence and Dependence}
\textbf{LI}:
A set of vectors $\in \mathbb{R}^n$ is Linearly Independent
(LI) if the vector equation
$x_1\textbf{v}_1+\dots+x_n\textbf{v}_n=\textbf{0}$ can only
be solved with the trivial solution $\textbf{x}=\textbf{0}$
\\

\noindent \textbf{LD}:
A set of vectors $\in \mathbb{R}^n$ is Linearly Dependent(LD) if 
there exist weights $(c_1,c_2,\dots,c_n)$, not all zero where
$\lbrace c_1\textbf{v}_1+c_2\textbf{v}_2+\dots+c_n\textbf{v}_n\rbrace=0$
\begin{itemize}
\item{$m>n=col>row=vectors>dimenstions$}
\item{If a vector is a scalar multiple of another in the set, LD}
\end{itemize}
\end{subsection}
\end{section}

\begin{section}{Linear Transformations and the Standard Matrix}
\begin{subsection}*{Theorems}
\textbf{Theorem 5}: If a matrix is m*n, then the transform 
$\textbf{x} \Rightarrow A\textbf{x}$ has these properties:
\begin{enumerate}
\item{$A(\textbf{u}+\textbf{v})=A\textbf{u}+A\textbf{v}$
Vector addition is preserved}
\item{$A(c\textbf{u})=cA\textbf{u}$} Scalar multiplication
	is preserved.
\end{enumerate}
\textbf{Theorem 10}: If T is a linear transformation
$\mathbb{R}^n \Rightarrow \mathbb{R}^m$ then there exists a matrix
A with dimensions m*n such that $T(\textbf{x})=A\textbf{x}$ for
all $\textbf{x}\in\mathbb{R}$ 
\\
$A= \begin{bmatrix} T(\textbf{e}_1) & \dots & T(\textbf{e}_n) 
\end{bmatrix}$ whose jth column is $T(\textbf{E}_j)$ where 
$\textbf{e}_j$ is the jth column of the identity matrix in 
$\mathbb{R}^n$
\\
\\ \textbf{Theorem 11}: Let T:$\mathbb{R}^n\Rightarrow\mathbb{R}^m$
be a linear transformation. Then T is one-to-one iff the equation $T(\textbf{x})=0$ has only
a trivial solution (LI)
\\
\\ \textbf{Theorem 12}:
Let T:$\mathbb{R}^n\Rightarrow\mathbb{R}^m$
be a linear transformation. A is the standard matrix of T
\begin{itemize}
\item{T maps $\mathbb{R}^n\Rightarrow\mathbb{R}^m$ iff columns of
	A span $\mathbb{R}^m$}
\item{T is one-to-one iff columns of A are LI}
\end{itemize}
\end{subsection}
	
\begin{subsection}*{Topics}
\begin{enumerate}
\item{Linear Transformations}
\item{Transformation Groups}
\item{One-to-one and Onto}
\end{enumerate}
\end{subsection}

\begin{subsection}{Linear Transformations}
Linear transformation allows for the mapping from one set
of vectors in a real space, to another set, in another
real space.\\
\textbf{Matrix Mapping}: The process of going from one set in one
space to another set in another space. A T transform is a function
of matrix mapping.\\
From this, a T from $\mathbb{R}^n \Rightarrow \mathbb{R}^m$ is a 
rule that assigns a vector $\textbf{x}\in\mathbb{R}^n$ to a vector
in $\mathbb{R}^m$.\\
\\ \textbf{Transforms $T:\mathbb{R}^n \Rightarrow \mathbb{R}^m$}
\begin{enumerate}
\item{Set $\mathbb{R}^n$ is in the \textbf{domain} of T}
\item{Set $\mathbb{R}^m$ is in the \textbf{codomain} of T}
\item{For an $\textbf{x}\in\mathbb{R}^n$, the vector 
$T\textbf{x}\in\mathbb{R}^m$ is called the \textbf{image} of 
\textbf{x} under the action T}
\item{The set of all images is called the \textbf{range} of T}
\end{enumerate}
T(\textbf{x}) formula: Linear Transformation
$\mathbb{R}^n \Rightarrow \mathbb{R}^m$ is actually a matrix
transformation $A\textbf{x}\Rightarrow\textbf{x}$. The key is to 
look at what T does to the columns of the identity matrix.
\end{subsection}

\begin{subsection}{Identity Matrices and Transformation Groups}
\textbf{Identity Matrix}: 
$\begin{bmatrix} 1 & 0 \\ 0 & 1 \end{bmatrix} 
\begin{bmatrix} \textbf{x}_1 \\ \textbf{x}_2 \end{bmatrix} 
= \begin{bmatrix} \textbf{x}_1 \\ \textbf{x}_2 \end{bmatrix} $
\\
\\ \textbf{Group 1}: Contraction and Expansion 
$\begin{bmatrix} K & 0 \\ 0 & 1 \end{bmatrix} \mbox{and} 
\begin{bmatrix} 1 & 0 \\ 0 & K \end{bmatrix}$
\\
\textbf{Group 2}: Reflection 
$\begin{bmatrix} 0 & 1 \\ 1 & 0 \end{bmatrix}$
\\
\textbf{Group 3}: Shearing
$\begin{bmatrix} 1 & K \\ 0 & 1 \end{bmatrix}
\begin{bmatrix} 1 & 0 \\ K & 1\end{bmatrix}  $
\\
\textbf{Group 4}: Rotations
$\begin{bmatrix} cos\Theta & -sin\Theta \\
sin\Theta & cos\Theta \end{bmatrix} $
\\
\textbf{Group 5}: Axis Projection
$\begin{bmatrix} 1 & 0 \\ 0 & 0 \end{bmatrix} 
\begin{bmatrix} 0 & 0 \\ 0 & 1 \end{bmatrix} $
\end{subsection}

\begin{subsection}{One-to-One and Onto}
\Large{\textbf{Onto}}: A mapping T: 
$\mathbb{R}^n\Rightarrow\mathbb{R}^m$ is \textbf{onto} if each
$\textbf{b}\in\mathbb{R}^m$ is the image of at least one
$\textbf{x}\in\mathbb{R}^n$ \\
\\ \noindent\Large{\textbf{One-to-one}}: A mapping 
T: $\mathbb{R}^n\Rightarrow\mathbb{R}^m$ is \textbf{one-to-one}
if each $\textbf{b}\in\mathbb{R}^m$ is the image of at most one
$\textbf{x}\in\mathbb{R}^n$
\end{subsection}
\end{section}

\newpage
\begin{section}{Matrix Multiplication and Invertability}
\begin{subsection}*{Theorems}
\textbf{Theorem 1}:
\begin{itemize}
\item{$(A^T)^T=A$}
\item{$(A+B)^T=A^T+B^T$}
\item{$(rA)^T=rA^T$}
\item{$(AB)^T=A^TB^T$}
\end{itemize} 
\textbf{Theorem 5}: If A is an n*n matrix and is invertible, then for 
every $\textbf{b}\in\mathbb{R}^n$ the equation $A\textbf{x}=\textbf{b}$
has a unique solution: $A^{-1}\textbf{b}=\textbf{x}$ \\
\noindent\textbf{Theorem 6}: If A is invertible, assume:
\begin{itemize}
\item{$(A^{-1})^{-1}=A$}
\item{$(AB)^{-1}=A^{-1}B^{-1}$}
\item{$(A^{T})^{-1}=(A^{-1})^{T}$}
\end{itemize}
\noindent\textbf{Theorem 7}: An n*n matrix $A\in\mathbb{R}^{n}$ is 
invertible iff A is row equivalent to $I_{n}$ \\
Any sequence of ERO's that transform $ A\Rightarrow I_{n}$ also 
transforms $I_{n} \Rightarrow A^{-1}$
\end{subsection} 
\begin{subsection}*{Topics}
\begin{enumerate}
\item{Matrix Properties and Definition}
\item{Row Column Computation}
\item{Invertibility}
\end{enumerate} \end{subsection}

\begin{subsection}{Matrix Properties and Definitions}
A,B,C are Matrices where m=n and r and s are scalars
\begin{enumerate}
\item{$A+B=B+A$}
\item{$A+0=A$}
\item{$r(A+B)=rA+rB$}
\item{$rs(A)=rA+sA$}
\end{enumerate}
A diagonal Matrix is a matrix where all non-diagonal entries are 0\\
A zero matrix is a matrix where all entries are 0\\
A+B adds each corresponding entry $(a+b)_{ij}=a_{ij}+b_{ij}$\\
Scalar multiplication is preserved \\
Matrix multiplication isn't commutative\\
\begin{itemize}
\item{$AB=AC\neq B=C$}
\item{$(AB)C=A(BC)$}
\item{$AB=0$ doesn't mean $A=0 or B=0$}
\item{$AB\neq BA$}
\end{itemize}
A is m*n, B and C have operable dimensions
\begin{itemize}
\item{$A(BC)=(AB)C$}
\item{$A(B+C)=AB+AC$ Left Distributive}
\item{$(B+C)A=BA+CA$ Right Distributive}
\item{$r(AB)=rA(B)=A)rB)$ Scalars}
\item{$I_{m}A=A=AI_{n}$}
\end{itemize}
\end{subsection}

\begin{subsection}{Row Column Computation}
$B\textbf{x}$ transforms $\textbf{x} \Rightarrow B\textbf{x}$\\
Matrix A turn $B\textbf{x}\Rightarrow A(B\textbf{x})$\\
For this to work, A must be m*n and B must be n*p. The outcome matrix
is n*p. Acol must equal Brow.
\\
\textbf{Row column rule for computing AB}: If AB is defined Acol=Brow
then the entry at row i and column j of AB is
\begin{equation}
	(AB)_{ij}=a_{i_1}b_{1j}+a_{i_2}b_{21}+\dots+a_{in}b_{nj}
\end{equation}
\end{subsection}

\begin{subsection}{Invertibility}
\begin{equation}
	A^{-1}A=AA^{-1}=I
\end{equation}
If $A^{-1}$ exists for A, A is invertible.\\
Non-invertible $\Leftrightarrow$ singular\\
$A^{-1}$ is unique\\
\textbf{The Elementary Matrix}: Obtained by performing a single ERO on an
identity matrix.\\
An ERO becomes an EA by performing operations on I.\\
The inverse of E is the elementary matrix of the same type that transforms
E back into I.\\ 
Apply ERO's both to A and I at the same time in one augmented matrix gives
$A^{-1}$ once A is in RREF.
\end{subsection}

\end{section}

\newpage
\begin{section}{Determinants and the Invertible Matrix Theorem}
\begin{subsection}*{Theorems}
\textbf{Theorem 2.8}: The Invertible Matrix Theorem (IMT) -- Let A be a
square n*n matrix.
\begin{enumerate}
\item{If any statements below are true, all are true. 
If any statements below are false all are false}
\begin{enumerate}
\item{A is an invertible matrix}
\item{A is row equivalent to the n*n identity matrix}
\item{A has n pivot positions}
\item{the equation $A\textbf{x}=\textbf{0}$ has only the trivial solution}
\item{The columns of A form a linearly independent set}
\item{The linear transformation $\textbf{x}\Rightarrow A\textbf{x}$ 
	is One-to-One}
\item{The equation $A\textbf{x}=\textbf{b}$ has at least one
	solution for each $\textbf{b}\in\mathbb{R}^n$}
\item{The columns of A span $\mathbb{R}^n$}
\item{The linear transformation $\textbf{x}\Rightarrow A\textbf{x}$
	maps $\mathbb{R}^n$ onto $\mathbb{R}^n$}
\item{There is an n*n matrix C such that CA=I}
\item{There is an n*n matrix D such that AD=I}
\item{$A^T$ is an invertible matrix}
\end{enumerate} \end{enumerate}
\textbf{Theorem 2.9}: Let T:$\mathbb{R}^n\Rightarrow\mathbb{R}^n$
be a linear transformation and let A be the standard matrix of T,
then T is invertible iff A is an invertible matrix.
\\
\\\textbf{Theorem 2.5}: let 
$A=\begin{bmatrix} a_{11} & a_{12} \\ a_{21} & a_{22} 
\end{bmatrix}$, if $ad-bd=0$ then A is invertible and 
\\$A^{-1}=\frac{1}{a_{11}a_{22}-a_{12}a_{21}}*
\begin{bmatrix} a_{22} & -a_{12} \\ -a_{21} & a_{11} \end{bmatrix}$\\
If $a_{11}a_{22}-a_{12}a_{21}=0$ then A is not invertible
detA$=a_{11}a_{22}-a_{12}a_{21}$
\\
\\ \textbf{Theorem 3.1}: The determinant of an n*n matrix A can be
computed by a cofactor expansion across any row or down any column.
\\ The cofactor expansion across the ith row:\\
$detA=a_{i1}c_{i1}+a_{i2}C_{12}+\dots+a_{in}C_{in}$\\
The Cofactor expansion down the jth column:\\
$detA=a_{1j}C_{1j}+a_{2j}C_{2j}+\dots+a_{nj}C_{nj}$
\\
\textbf{Theorem 3.2}: If A is a triangular matrix, then detA is the 
product of the entries on the main diagonal
\end{subsection}
\begin{subsection}{Topics}
\begin{enumerate}
\item{Invertability, Span, LI, Pivots, in the context of the
	Invertible Matrix Theorem (IMT)}
\item{Determinants of 3x3 matrices and NxN matrices}
\end{enumerate}
\end{subsection}

\begin{subsection}{Determinants}
\begin{equation}
	a_{11}\neq0\\
\begin{bmatrix} a_{11} & a_{12} & a_{13} \\ a_{21} & a_{22} & a_{23}
\\ a_{31} & a_{32} & a_{33} \end{bmatrix} \Rightarrow
\begin{bmatrix} a_{11} & a_{12} & a_{13} \\
0 & a_{11}a_{22}-a_{12}a_{21} & a_{11}a_{23}-a_{13}a_{21} \\
0 & 0 & a_{11}\Delta \end{bmatrix} \\ \\
\end{equation}
\begin{equation}
\Delta=a_{11}a_{22}a_{33}+a_{12}a_{23}a_{31}+a_{12}a_{21}a_{32}-
a_{11}a_{23}a_{31}-a_{12}a_{21}a_{33}-a_{13}a_{22}a_{31}
\end{equation}
\begin{equation}
A invertible \Leftarrow\Rightarrow \Delta\neq0 \Rightarrow detA=\Delta
\end{equation}
\begin{equation}
\Delta = a_{11}det\begin{bmatrix} a_{22} & a_{23}
\\ a_{32} & a_{33} \end{bmatrix} 
-a_{12}det\begin{bmatrix} a_{21} & a_{23} \\ a_{31} & a_{33} 
\end{bmatrix} 
+a_{13}det\begin{bmatrix} a_{21} & a_{22} \\
a_{31} & a_{32} \end{bmatrix} 
\end{equation}

\noindent\textbf{Determinant Definition}: For n=2, 
the determinant of an n*n
matrix A is \\ 
\begin{equation}
detA=a_{11}detA_{11}-A_{12}detA_{12}+\dots+(-1)^{1+n} a_{1n}detA_{1n}=
\sum_{j=1}^{n}(-1)^{1+j}a_{ij}detA_{ij}
\end{equation}
\\ \\
\noindent\textbf{Cofactor}: Define th(ij) - cofactor of A as the number 
$C_{ij}$
\begin{equation}
	C_{ij}=(-1)^{i+j}detA_{ij} \Rightarrow
	detA= a_{11}c_{11}+a_{12}c_{12}+\dots+a_{1n}c_{1n}
\end{equation}
The cofactor expansion across the 1st row of A
\begin{equation}
detA=a_{11}C_{11}+a_{12}C_{12}+\dots+a_{1n}C_{1n}	
\end{equation}
The cofactor expansion across the 1st column of A
\begin{equation}
	detA=a_{11}C_{11}+a_{21}C_{21}+\dots+a_{n1}C_{n1}	
\end{equation}
\textbf{Triangular Matrix}: A matrix that has all zeros
either below or above the main diagonal. \\
Upper triangular: 0's below the diagonal\\
Lower Triangular: 0's above the diagonal
\end{subsection}

\begin{subsection}{IMT Patterns}
$a\Rightarrow j\Rightarrow d\Rightarrow c\Rightarrow b\Rightarrow 
a\\
a\Rightarrow k\Rightarrow g\Rightarrow a\\
g\Leftrightarrow h\Leftrightarrow i \\
d\Leftrightarrow e\Leftrightarrow f \\
a\Leftrightarrow l$
\\
See Theorem 8 for more
\end{subsection}
\end{section}

\newpage
\begin{section}{Determinant properties and Cramer's Rule}
\begin{subsection}*{Theorems}
\textbf{Theorem 3.3}: Row operations:\\
Let A be a square matrix. 
\begin{enumerate}
\item{If a multiple of one row of A is added to another row to produce
a matrix B (Replacement ERO), the detB=detA}
\item{If two rows of A are interchanged to produce B, then detB=-detA}
\item{If one row of A is multiplied by K to produce B, then detB=KdetA}
\end{enumerate}
\textbf{Theorem 3.5}: if A is an n*n matrix, $det(A^{T})=detA$\\
\noindent\textbf{Theorem 3.6}: Multiplicative Property. If A and B are n*n
then
\begin{equation}
det(AB)=detAdetB
\end{equation}

\noindent\textbf{Theorem 3.7 (Cramer's Rule}: Let A be an invertible 
matrix, for any $\textbf{b}\in\mathbb{R}^{n}$, the unique solution
\textbf{x} of A\textbf{x} has entries given by $x_{i}=\frac{detA_{i}
\textbf{b}}{detA}$\\
The Proof of this is shown by:
\begin{equation}
A=\begin{bmatrix} \textbf{a}_{1}, \textbf{1}_{2},\dots,\textbf{x},
\dots, \textbf{a}_{n} \end{bmatrix} \Rightarrow AI_{i}\textbf{x}=
\begin{bmatrix} A\textbf{e}_{1},A\textbf{e}_{2},\dots,A\textbf{x} 
,\dots,A\textbf{e}_{n} \end{bmatrix} 
\end{equation} Which, through the det multiplicative property gives us
\begin{equation}
	detA(det(I_{i}(\textbf{x})))=det(A_{i}(\textbf{b})
\end{equation} \\
\noindent\textbf{Theorem 3.9}: If A is a 2x2 matrix, the area of the
paralleogram determined by the columns of A is |detA| \\
\noindent if A is a 3x3 matrix, the volume of the parallelepiped determined
by the columns of A is $\lvert detA\rvert$ 
\\
\noindent\textbf{Theorem 3.10}: Let T: $\mathbb{R}^{2}\Rightarrow \mathbb{R}^{2}$ 
be a linear transformation by a 2*2 matrix A, if S is a 
parallelogram in $\mathbb{R}^{2}$, then \\
Area of $T(S) = |detA|(Area of S)$ \\
If T: $\mathbb{R}^{3}\Rightarrow \mathbb{R}^{3}$ is determined by a 3*3
matrix A, S is a parallelepiped in $\mathbb{R}^{3}$\\
Volume of T(S) = |detA|(Volume of S)
\end{subsection}
\begin{subsection}*{Topics}
\begin{enumerate}
\item{understand how ERO's affect the determinants, and how to 
use this (together with determinants of diagonal matrices) 
to compute the determinant}
\item{Cramer's Rule for solving systems of equations}
\item{Properties of Determinants}
\end{enumerate}
\end{subsection}
\begin{subsection}{Determinant Operations}
Using Theorem 3, ERO's to attain a diagonal matrix can get the determinant.
\begin{equation}{Example:}
\begin{bmatrix} 0&1&5 \\ 3&-6&9\\ 2&6&1 \end{bmatrix} \Rightarrow
\begin{bmatrix} 3&-6&9\\ 0&1&5 \\ 2&6&1 \end{bmatrix} \Rightarrow
\begin{bmatrix} 1&-2&3\\ 0&1&5\\ 0&0&-55 \end{bmatrix}\Rightarrow 
detA=1*1*(-55)*(-1)*3
\end{equation}
Suppose a matrix A has been reduced to echelon form U by row replacement 
operations and row interchanges.\\
If there are r total interchanges then THM 3 says:
\begin{equation} DetA=(-1)^{r}detU \end{equation}
Since U is in Echelon Form, (triangular), detU is the product of the
diagonal entries $u_{11},u_{22},\dots,u_{nn}$\\
If A is invertible, the entries on the diagonal are all pivots
$A~I_{n}$\\
If A is non-invertible, at least one of these entries of the row reduced
form $u_{11},u_{22},\dots,u_{nn}$ must be 0, and the determinant must
also be equal to 0.
\begin{equation}{When~A~is~invertible:}
detA=(-1)^{r}*(products~of~the~pivots~of~U) \end{equation}
\begin{equation}{When~A~is~singular:}detA=O\end{equation}
\end{subsection}

\begin{subsection}{Column Replacement for Cramer's}
Look at Theorem 3.7\\
For any n*n matrix A and any $\textbf{b}\in\mathbb{R}^{n}$, let
\textbf{b} be the matrix obtained from A by replacing column i by the 
vector \textbf{b}.
$A_{i}\textbf{b}=\begin{bmatrix} \textbf{a}_{1}, \textbf{a}_{2},\dots,
\textbf{b},\dots, a_{n} \end{bmatrix}$
\end{subsection}
\begin{subsection}{Geometrical view of Determinants}
Look at Theorem 3.9
Example of how this works:
\begin{equation}
\lvert det\begin{bmatrix} a&0\\0&d \end{bmatrix} \rvert = 
\lvert ad \rvert = A~of~a~rectangle
\end{equation}
Convert points to starting at (0,0) and get the determinant from there. \\
The conclusion of Theorem 10 holds whenever S is a region $\in\mathbb{R}^2$
with finite area or a region $\in\mathbb{R}^{3}$ with finite volume. \\

\end{subsection}
\end{section}
\newpage

\begin{section}{Vector Spaces and Kernels}
\begin{subsection}*{Theorems}
\textbf{Theorem 4.1}: If $\textbf{v}_{1},\dots,\textbf{v}_{p}$ are in a
vector space V.\\
Then, Span$\{\textbf{v}_{1},\dots,\textbf{v}_{p}\}$
is a subspace of V\\
\\
\textbf{Theorem 4.2}: The Null space of an m*n matrix A is a subspace
of $\mathbb{R}^{n}$\\
\textbf{Proof}:
\begin{itemize}
\item{$\textbf{0}\in NulA$ because $A\textbf{0}=\textbf{0}$}
\item{$\textbf{u},\textbf{v}\in NulA \Rightarrow A\textbf{u}-\textbf{0},
A\textbf{v}=\textbf{0}\Rightarrow A\textbf{u}+A\textbf{v}=\textbf{0}$}
\item{$\textbf{u}\in NulA \Rightarrow A\textbf{u}=cA\textbf{u}=0$}
\end{itemize}
\textbf{Theorem 4.3}:
The column space of an m*n Matrix A is a subspace of $\mathbb{R}^{m}$
\\
\textbf{Theorem 4.4}: An indexed set (Put in a certain order) of 2 or
more vectors $\{\textbf{v}_{1}, \dots, \textbf{v}_{p}\}$ with 
$\textbf{v}_{1}\neq 0$, is linearly dependent iff some $\textbf{v}_{j}$ 
$j>1$ is a linear combination of the vectors $\textbf{v}_{1},\dots,
\textbf{v}_{j-1}$
\\ \noindent\textbf{Theorem 4.5}: Let $S=\{\textbf{v}_{1},\dots,
\textbf{v}_{p}\}$ be set in V and let $H=\mbox{ Span}\{
\textbf{v}_{1},\dots,\textbf{v}_{p}\}$
\begin{itemize}
\item{If one of the vectors in S, denoted $\textbf{v}_{k}$ is a 
linear combination of the remaining vectors in S, then the set
 formed from S by removing $\textbf{v}_{k}$ still span H}
\item{If $H=\{\textbf{0\}}$ (set of just the \textbf{0} vector)
, some subset of S is a basis of H} 
\end{itemize}
\noindent\textbf{Theorem 4.6}: The Pivot columns of Matrix A form a basis for ColA
\end{subsection}
\begin{subsection}*{Topics}
\begin{enumerate}
\item{Vector Spaces}
\item{Null Spaces}
\item{Column Spaces}
\item{Kernel}
\item{Bases}
\end{enumerate}
\end{subsection}
\begin{subsection}{Vector Spaces}
\textbf{Vector Space}: A vector space is a non-empty set V of objects,
called vectors, on which are defined two operations, called addition
and multiplication by scalars ($\mathbb{R}$) subject to the 10 axioms
listed below. The Axioms must hold for all the vectors \textbf{u},
\textbf{w}, and \textbf{v} in V and for all scalars c and d
\begin{enumerate}
\item{The sum of \textbf{u} and \textbf{v}, denoted by,
$\textbf{u}+\textbf{v}$, is in V}
\item{$\textbf{u}+\textbf{v}=\textbf{v}+\textbf{u}$}
\item{$(\textbf{u}+\textbf{v})+\textbf{w}=
\textbf{u}+(\textbf{v}+\textbf{w})$}
\item{There is a zero vector \textbf{0} in V such that 
$\textbf{u}+0=\textbf{u}$}
\item{For each \textbf{u} in V, there is a vector -\textbf{u} in V
such that $\textbf{u}+(-\textbf{u}=\textbf{0}$}
\item{The scalar multiple of \textbf{u}by c, denoted by c\textbf{u}, is
in V}
\item{$c(\textbf{u}+\textbf{v})=c \textbf{u}+c \textbf{v}$}
\item{$(c+d)\textbf{u}=c\textbf{u}+d\textbf{u}$}
\item{$c(d\textbf{u})=(cd)\textbf{u}$}
\item{$1\textbf{u}=\textbf{u}$}
\end{enumerate}
$P_{n}$ for $n \geq 0$, polynomials with degree of at most n
$P(t)=a_0+a_1t+a_2t^{2}+\dots+a_{n}t^{n}$\\
$q(t)=b_1t+b_2t^{2}+\dots+b_{n}t^{n}$\\
Addition:
$(p+q)(t)=p(t)+q(t)$
Scalar Multiplication:
$cp(t)=cp(t)=ca_0+ca_1n+\dots+ca_{n}^{n}$
\\ \\
\textbf{Subspaces:} A subspace of vector space V is a subset H of V that 
has three primary properties\\
\begin{enumerate}
\item{The zero vector of V is in H}
\item{H is closed under vector addition $\Rightarrow \textbf{u},
\textbf{v} \in H, \textbf{u}+\textbf{v}\in H$}
\item{H is closed under multiplication by scalars\\
For any \textbf{u} in H , any scalar c, $c\textbf{u}\in H$}
\end{enumerate}
\begin{equation}
\Biggl\{ \begin{bmatrix} s \\ t \\ 0 \end{bmatrix} 
\mbox{: s and t are real} \Biggl\}
\end{equation}
All three aspects of the subspace definition are met
\begin{enumerate}
\item{$\textbf{0}=\begin{bmatrix} 0\\ 0\\ 0 \end{bmatrix} \in H$}
\item{$\begin{bmatrix} s_1\\ t_1\\ 0 \end{bmatrix} 
\begin{bmatrix} s_2\\ t_2\\ 0 \end{bmatrix}= 
\begin{bmatrix} s_1+s_{2}\\ t_1+t_2\\ 0 \end{bmatrix}$}
\item{$c(\begin{bmatrix} s\\ t\\ 0 \end{bmatrix}) = \begin{bmatrix} 
cs\\ ct\\ 0\end{bmatrix} \in H$}
\end{enumerate}
\end{subsection}
\begin{subsection}{Null Space}
\begin{equation}
\begin{cases}
	x_1-3x_2-2x_3=0 \\
	-5x_1+9x_2+x_3=0\\
\end{cases}
\end{equation}
\begin{equation}
\Rightarrow \begin{bmatrix} 1&-3&-2\\ -5&9&1 \end{bmatrix} 
\Rightarrow
\textbf{x}=\begin{bmatrix} x_1\\ x_2\\ x_3 \end{bmatrix} 
\end{equation}
We call the set of x that satisfy $A\textbf{x}=\textbf{0}$ the 
null space of A.
\\ \noindent\textbf{Definition:} The null space of an m*n matrix A,
denoted by NulA is the set of all solutions of the homogenous equation 
$a\textbf{x}= \textbf{0}$\\
NulA $=\{ \textbf{x} \mbox{ is in } \mathbb{R}^{n} \mbox{and} 
A\textbf{x}=\textbf{0}\}$
\end{subsection}

\begin{subsection}{Column Space}
\noindent \textbf{Definition}: 
The Column Space of an m*n matrix A (ColA) is the set of all linear 
combinations of the columns of A, $A=\begin{bmatrix} 
\textbf{a}_{1}, \textbf{a}_{2},\dots, \textbf{a}_{n} \end{bmatrix}$
\begin{equation}
\begin{cases}
\mbox{ColA}=Span\{\textbf{a}_{1},\dots, \textbf{a}_{n}\} \\
\mbox{ColA} = \{ \textbf{b}: \textbf{b}=A\textbf{x} \mbox{ for } 
	\textbf{x} \in\mathbb{R}^{n}\}
\end{cases}
\end{equation}
\end{subsection}
\begin{subsection}{Kernel}
A linear transformation T from a vector space V into a vector space W
is a rule that assigns each vector $\textbf{x}\in V$ to a unique vector
$T(\textbf{x})$ in w such that:
\begin{itemize}
\item{$T(\textbf{u}+\textbf{v}=T(\textbf{u}+T\textbf{v}$ for all
	$\textbf{u},\textbf{v}\in V$}
\item{$T(c\textbf{u})=cT(\textbf{u})$ for all \textbf{u} in V and 
	scalars c}
\item{The Kernel (or null space) is the set of all \textbf{u} in V such
that $T(\textbf{u})=\textbf{0}$} with the kernel a subspace of V
\item{The range of T is the set of all vectors in W of the form
	T(\textbf{x}) for BLANK is a subspace of W.}
\end{itemize}
\end{subsection}

\begin{subsection}{Bases}
\textbf{Basis Definition}: Let H be a subspace of vector space V,
an indexed set of vectors
$B=\{\textbf{b}_{1},\dots,\textbf{b}_{p}\}$ in V is a basis for H if
\begin{itemize}
\item{B is a linearly independent set}
\item{$H=\mbox{Span}\{\textbf{b}_{1},\dots,\textbf{b}_{p}\}$}
\end{itemize}
\end{subsection} \end{section}

\newpage
\begin{section}{Dimension in Vector Spaces and Rank}
\begin{subsection}*{Theorems}
\textbf{Theorem 4.9}: If a vector space V has a basis B
$B=\{\textbf{b}_{1},\dots,\textbf{b}_{n}\}$, then a set in V 
containing more than n vectors must be linearly dependent\\
\noindent\textbf{Theorem 4.10}: If a vector space V has a basis of
n vectors, then every basis of V must consist of exactly n vectors.
\\ \textbf{Proof}: let $B_1$ be a basis of n vectors and $B_2$ 
be another basis of V. Since $B_1$ is a basis and $B_2$ 
is linearly independent $B_2$ has no more than n vectors, by THM 9.
Since $B_2$ is a basis and $B_1$is linearly independent, $B_2$ 
has at least n vectors. Thus $B_2$ consists of exactly n vectors.
\\ \noindent \textbf{Theorem 4.11}:  Let H be a subspace of a finite
dimensional vector space V, any LI set in H can be expanded, if 
necessary, to form a basis of H. Also, H is finite dimensional and
$DimH \leq DimV$
\\ \noindent \textbf{Theorem 4.12}: The Basis Theorem\\
\begin{itemize}
\item{let V be a p-dimensional vector space, $v\geq 1$}
\item{Any linearly independent set of exactly p elements in V is automatically a 
	basis for V}
\item{Any set of p elements that spans V is automatically a basis for V}
\end{itemize}
\noindent \textbf{Theorem 4.13}: if 2 matrices A and B are row equivalent, then 
their row spaces are the same. \\
If B is in echelon form, the non-Zero rows of B form a basis for the row space of A
as well as for that of B. \\
\noindent \textbf{Theorem 4.14 (The Rank Theorem)}:
\begin{itemize}
\item{The dimension of the column space and the row space of an m*n matrix A are equal}
\item{This common dimension, the rank of A, also equals the number of pivot positions
in A and satisfies: $\mbox{rank}A+\mbox{dim NulA}=n$} 
\end{itemize}
\noindent \textbf{New IMT}: Let A be n*n, A is invertible, also applies to $A^{T}$
\begin{itemize}
\item{The Columns of A form a basis of $\mathbb{R}^{n}$}
\item{$ColA=\mathbb{R}^{n}$}
\item{$dimColA=n$}
\item{$rankA=n$}
\item{$NulA=\{0\}$}
\item{dimNulA$=0$}
\end{itemize}
\end{subsection}
\begin{subsection}*{Topics}
\begin{enumerate}
\item{Dimensions}
\item{Row Space}
\end{enumerate}
\end{subsection}
\begin{subsection}{Dimensions}
\textbf{Dimension Definition}:	If V is spanned by a finite set,
then V is said to be finite dimensional and the dimension of V,
written as DimV is the number of vectors in a basis for V. If V is
not spanned by a finite set, then V is of infinite dimensions.\\
\\ The Standard basis of $\mathbb{R}^{n}$ contains n vectors.
\begin{equation}
\textbf{e}_{1}=\begin{bmatrix} 1\\ 0 \\ \dots \\ 0 \end{bmatrix}
\textbf{e}_{2}= \begin{bmatrix} 0\\ 1\\ 0\\ \dots \\ 0 \end{bmatrix}
\textbf{e}_{n}= \begin{bmatrix} 0\\ 0\\ 0\\ \dots \\ n \end{bmatrix}
\end{equation}
The standard polynomial basis $\{1,t,t^{2}\}$ spans P2. dimP2=3
Let $H=Span\{\textbf{v}_1,\textbf{v}_2$ where $\textbf{v}_{1}$\\
The subspaces of $\mathbb{R}^{3}$ can be classified by dimension.\\
\begin{itemize}
\item{0-dimensional: only the zero subspace
$\begin{bmatrix} 0\\ 0\\ 0 \end{bmatrix}$ }
\item{1-dimensional: Any subspace defined by a single nonzero vector
	with lines through the origin}
\item{2-dimensional: spanned by two LI vectors (A plane through the 
	origin}
\item{3-dimensional: Only $\mathbb{R}^{3}$ itself. Any 3 LI vectors 
in $\mathbb{R}^{3}$ span $\mathbb{R}^{3}$}
\end{itemize}
The pivot columns of a matrix A form a basis for ColA $A=[ \textbf{a}_{1}, \dots, 
\textbf{a}_{n}]$ \\
Then we know the dimensions of ColA as span as we know the pivot columns. Finding the 
dimensions of NulA seems like it would take more, but there's a shortcut. \\ \\
Assume A is m*n and $A\textbf{x}=0$ has k free variables. The method for producing a
basis for NulA will produce exactly k linearly independent vectors.\\
\begin{itemize}
\item{dim ColA is the number of pivot columns of A}
\item{dim NulA is the number of free variables in the equation $A\textbf{x}=0$}
\end{itemize}
\begin{equation}
\{\mbox{RankA (number of pivot cols)}\} + \{\mbox{dimNulA (number of non-pivot cols)}\}
=\{\mbox{n (total number of columns)}\}
\end{equation}
\end{subsection}
\begin{subsection}{Row Space}
\textbf{Definition}: The row space is the set of all linear combinations of the
row vectors.\\
The rows of A correspond to the columns of $A^{T} \Rightarrow \mbox{RowA=Col}A^{T}$
\\ If we know a linear dependence relation on any of the rows of A, we could use
the Spanning Set Theorem to shrink the set to a basis
\end{subsection}
\begin{subsection}{Rank}
\textbf{Definition}: The rank of A is the dimension of the column space of A
\begin{equation}
\mbox{RankA}=\mbox{dim ColA}
\end{equation}
Since $RowA=ColA^{T}$, $\mbox{dim RowA}=\mbox{rank}A^{T}$ \\
Look at THM 14 and 15
\end{subsection}
\end{section}
\newpage
\begin{section}{Coordinate systems}
\begin{subsection}*{Theorems}
\textbf{Theorem 5.7 - The Unique Representation Theorem}:\\
Let $B=\{\textbf{b}_{1},\dots,\textbf{b}_{n}\}$ be a basis for a vector space V.\\
Then for each $\textbf{x} \in V$, there exists a unique set of scalars such that
$\textbf{x}=c_1\textbf{b}_1+\dots+c_{n}\textbf{b}_{n}$\\
Existence: Since B spans V there exists scalars $c_1,\dots,c_{n}$ such that 
$\textbf{x}=c_1\textbf{v}_{1}+\dots+c_{n}\textbf{v}_{n}$\\
Uniqueness: Assume there exists others scalars $d_1,\dots,d_{n}$ such that 
$\textbf{x}=d_1\textbf{b}_{1}+\dots+d_{n}\textbf{b}_{n}$\\
Since B is LI, each of the coefficients $c_{k}-d_{k}=0$, Thus c=d for all entries.
\\ \noindent \textbf{Theorem 5.8}: Let $\beta=\{\textbf{b}_{1},\dots,\textbf{b}_{n}$
be a basis for a vector space V.\\
Then the coordinate mapping $\textbf{x}\Rightarrow [\textbf{x}_{B}$ is a
one-to-one linear transformation from V onto $\mathbb{R}^{n}$ (This transform is 
both onto and one-to-one). \\
Proof:
\begin{enumerate}
\item{Demonstrate the coordinate transform is linear}
\item{Show the transform is one-to-one (uniqueness)}
\item{Show the transform is onto (existence)}
\end{enumerate}
\noindent \textbf{Theorem 5.15}: Let $\beta=\{\textbf{b}_{1},\dots,\textbf{b}_{n}
\} ~\& ~ C=\{\textbf{c}_{1},\dots,\textbf{c}_{n}\}$ be bases of a vector space V.\\
Then there is a unique n*n matrix $P_{C\gets B}$ such that 
\begin{equation}
	[\textbf{x}]_{C}=P[\textbf{x}]_{B}
\end{equation}
The columns of $P C \Leftarrow B$ are the C-coordinate of the vectors in the 
basis B, change-of-coordinates matrix from B to C 
\end{subsection}
\begin{subsection}*{Topics}
\begin{itemize}
\item{Basis Coordinates}
\item{Coordinate Systems}
\item{Transformations Between Spaces}

\end{itemize}
\end{subsection}
\begin{subsection}{Basis Coordinates}
Definition: Suppose $B=\textbf{b}_{1,\dots,\textbf{b}_{n}}$ is a basis for V and 
$x \in V$. The coordinates of x relative to B (or the B coordinates of \textbf{x})
are the weights $c_1,\dots,c_{n}$ such that $\textbf{x}=c_1\textbf{b}_{1}+\dots+
c_{n}\textbf{b}_{n}$
\begin{equation}
[x]_{b}=\begin{bmatrix} c_1\\ c_2\\ \dots\\ c_{n} \end{bmatrix} 
\end{equation}
The coordinate vector of \textbf{x} relative to $B_1$ or the B-coordinate vector 
of \textbf{x}. \\
The Coordinate mapping $\textbf{x}\Rightarrow [x]_{B}$ is the coordinate mapping 
determined by B.\\
Standard Basis: $\varepsilon = \{\textbf{e}_1,\textbf{e}_{2}\}$
\\ The Matrix $[\textbf{b}_1,\textbf{b}_{2}]=\begin{bmatrix} -1&3\\2&1
\end{bmatrix} $ changes the B-coordinates of a vector $\textbf{x}(c_1,c_2)$ for
\textbf{x}. \\
General Basis: $B=[\textbf{b}_{1},\textbf{b}_{2},\dots,\textbf{b}_{n}]$\\
Let: $P_{B}=[\textbf{b}_{1},\textbf{b}_{2},\dots,\textbf{b}_{n}]$ then the vector
equation $\textbf{x}=c_1\textbf{b}_{1}+\dots+c_{n}\textbf{b}_{n}$ is equivalent
to $P_{B}[\textbf{x}]_{B}=\textbf{x}$.\\
$P_{B}$ is called the change-of-coordinates matrix from B to the standard basis
in $\mathbb{R}^{n}$ the columns of $P_{B}$ for a basis for $\mathbb{R}^{n}$, so 
$P_{B}$ is invertible by the IMT $P_{B}^{-1}\textbf{x}=[\textbf{x}]_{B}$\\
By defining a basis in a vector space V, we can create a coordinate system to 
relate to $\mathbb{R}^{n}$.
\end{subsection}
\begin{subsection}{Coordinate systems}
\begin{itemize}
\item{A basis B creates a coordinate system for a vector space}
\item{Can map abstract vector spaces to $\mathbb{R}^{n}$}
\item{Different coordinate systems in $\mathbb{R}^{n}$ offer different views of 
	vector spaces}/
\end{itemize}
\textbf{Definition - Isomorphism}: Isomorphisms are transformations that are both one-to-one
and onto.\\
Let $\beta$ be the standard matrix for $\mathbb{P}_{3}$ of polynomials $p(t)=a_0+
a_1t+a_2t^{2}+a_3t^{3} \to \beta=\{1,t,t^{2},t^{3}\}$\\
\begin{equation}
[P]_{B}=\begin{bmatrix} a_0\\a_1\\a_2\\a_3 \end{bmatrix} \in \mathbb{R}^{4}
\end{equation}
\\ $\beta =\{\textbf{v}_{1},\textbf{v}_{2}\}$ is a basis for 
$H=Span\{\textbf{v}_{1},\textbf{v}_{2}\}$.\\ 
The coordinate vector of \textbf{x} relative to B is $[\textbf{x}]_{B}$
It can be worthwhile to look at vectors in different coordinate systems. Consider
a Vector \textbf{x} and bases B and C, which relate\\ 
$[\textbf{x}]_{B} \to [\textbf{x}]_{C} \mbox{ where } B=\{\textbf{b}_{1},
\textbf{b}_{2}\} \& C=\{\textbf{c}_{1},\textbf{c}_{2}\}$\\ 
for a vector space V such that 
$\textbf{b}_{1}=4\textbf{c}_{1}+\textbf{c}_{2} ~\& ~\textbf{b}_{2}=
-6\textbf{c}_{1}+\textbf{c}_{2}$ \\ Suppose:\\
\begin{equation}
[\textbf{x}]_{B}=\begin{bmatrix} 3\\1 \end{bmatrix} 
[\textbf{x}]_{C}=[[\textbf{b}_{1}]_{c}~[\textbf{b}_{2}]_{c}]\begin{bmatrix} 3\\1
\end{bmatrix} \to [\textbf{x}]_{C}=\begin{bmatrix} 4&-6\\ 1&1 \end{bmatrix} =
\begin{bmatrix} 6\\4 \end{bmatrix} 
\end{equation}
\end{subsection}
\begin{subsection}{Transformations Between Spaces}
Multiplication on the left by $P_{C\gets B}$ convert B-card to C-Card.\\
$P_{C\gets B}$ is square and has LI columns so it's invertible by the IMT.\\ 
\begin{equation}
	(P_{C \gets B})^{-1}[\textbf{x}]_{C}=[\textbf{x}]_{B}
\end{equation}
$B=\textbf{b}_{1}+\dots+\textbf{b}_{n} ~\& ~ \varepsilon \{\textbf{e}_{1}+\dots
+\textbf{e}_{n}\}.$ \\
By this, $[\textbf{b}_{1}]_{\varepsilon}=\textbf{b}_{1}$ and
$P_{\varepsilon \gets B}$ is $P_{B}$
\end{subsection}
\end{section}
\newpage
\begin{section}{Eigenvectors and Eigenvectors}
\begin{subsection}*{Theorems}
\textbf{Theorem 5.1}: The eigenvalues of a triangular matrix are the diagonal entries
\\ \noindent \textbf{Theorem 5.2}: If $\{ \textbf{v}_{1},\dots,\textbf{v}_{r}\}$ are 
eigenvectors that correspond to distinct eigenvalues $\lambda_{1},\dots, \lambda_{r}$
of an n*n matrix A, then the set $\{ \textbf{v}_{1},\dots,\textbf{v}_{r}\}$ is
Linearly Independent.\\
\noindent \textbf{IMT continued}: Let A be an n*n matrix. Then A is invertible iff \\
the number 0 is not an eigenvalue of A. (in addition to IMT properties from earlier)
\noindent \textbf{Theorem 5.4}: If we have two n*n matrices A and B, and they are
 similar, then they have the same characteristic polynomial and the same eigenvalues.
\\ \textbf{Proof}: If $B=P^{-1}AP$ then $B-\lambda I = P^{-1}AP-\lambda 
P^{-1}P=P^{-1}(AP-\lambda P)=P^{-1}(A-\Lambda I)P$. \\
\begin{equation}
(det(B-\lambda I)=det[P^{-1}(A-\lambda I)P]=det(P^{-1})det(A-\lambda I)det(P)=
det(A-\lambda I)
\end{equation}
\end{subsection}
\begin{subsection}*{Topics}
\begin{enumerate}
\item{Eigenvectors}
\item{Characteristic Equation}
\item{Similar Matrices}
\end{enumerate}
\end{subsection}
\begin{subsection}{Eigenvectors}
The standard process of transformation is $\textbf{x} \to A\textbf{x}$ which can 
move vectors in a variety of ways.\\ 
There are special vectors, where the action of A is quite simple. Said A's can scale
vectors despite it being a matrix transform - the matrix acts as a scalar. \\
E.g. $A\textbf{x}=2\textbf{x},~A\textbf{x}=4\textbf{x}$:
\textbf{Eigenvector Definition}: An eigenvector of an n*n matrix A is a nonzero
vector such that $A\textbf{x}=\lambda \textbf{x}$ for a scalar $\lambda$.\\
$\lambda$ is an eigenvalue of A if there is a nontrivial solution \textbf{x} of 
$A\textbf{x}=\lambda \textbf{x}$ with the eigenvector corresponding to $\lambda$\\
To prove that a $\lambda$ is an eigenvalue of a given matrix A, set $A\textbf{x}=
\lambda\textbf{x}$ and solve for \textbf{x}.\\
The \textbf{Eigenspace} is Span $\{[\textbf{x}]\}$\\
Row reduction can be used to find eigenvectors, but cannot help us find eigenvalues.
\\ In general, $\lambda$ is an eigenvalue of A iff:
\begin{equation} \label{Eigenvalues}
	(A-\lambda )\textbf{x}=\textbf{0}
\end{equation}
is a nontrivial solution. \\
\textbf{Eigenspace}: the set of all solutions is the null space of of the matrix 
$(A-\lambda I)$\\
This space is a subspace of $\mathbb{R}^{n}$ called the eigenspace of A corresponding
to $\lambda$. \\
To find the possible $\lambda$ of a given A, find det(A-$\lambda$I) such that it is
equal to zero.\\
\end{subsection}
\begin{subsection}{Characteristic Equation}
A scalar $\lambda$ is an eigenvalue of an n*n matrix A iff $\lambda$ satisfies the
scalar equation 
\begin{equation}
det(A-\lambda I)=0
\end{equation}
which we call the characteristic equation.\\
If A is an n*n matrix, then $det(A-\lambda I)$ is a polynomial of degree n, called
the characteristic polynomial. The algebraic multiplicity of an eigenvalue is its
multiplicity as a root of the characteristic equation.
\end{subsection}
\begin{subsection}{Similar matrices}
\textbf{Definition}: A is similar to B iff there is an invertible matrix P such that 
$P^{-1}AP=B$ or, equivalently, $A=PBP^{-1}$.\\
Write $Q=P^{-1} \to Q^{-1}BQ=A$. So, B is similar to A. A and B are similar. Goes both
ways. Changing A into $P^{-1}AP$ is a similarity transformation
\end{subsection}
\end{section}

\newpage
\begin{section}{Diagonalization}
\begin{subsection}*{Theorems}
\textbf{Theorem 5.5 "Diagonalization Theorem"}: an n*n matrix A is diagonalizable iff
A has n linearly independent eigenvectors. \\
$A=PDP^{-1}$ with a diagonal matrix D, iff the columns of P are n linearly independent
eigenvectors of A.\\
The diagonal entries of D are eigenvalues of A that correspond respectively to the
eigenvectors of P.
\noindent \textbf{Theorem 5.6}: An n*n matrix with n distinct eigenvalues is 
diagonalizable (proved by Theorem 5.2) \\
\noindent \textbf{Theorem 5.7}: Let A be an n*n matrix whose distinct eigenvalues are
$\lambda_{1}, \dots, \lambda_{p} \to p \in n$ \\
For $1 \leq k \leq p$, the dimension of the eigenspace for $\lambda_{k}$ is less than
or equal to the multiplicity of $\lambda_{k}$. \\
The Matrix A is diagonalizable iff the sum of the dimensions of the eigenspaces=n\\
\begin{itemize}
\item{The characteristic polynomial factors completely}
\item{The dimension of the eigenspace for each $\lambda_{k}$ = multiplicity of
$\lambda_{k}$}
\item{if A is diagonalizable and $B_{k}$ is a basis for the eigenspace for each k,
then the total collection of basis vectors is a basis for $\mathbb{R}^{n}$}
\noindent \textbf{Theorem 5.8 "Diagonal Matrix Representation}: Suppose $A=PDP^{-1}$,
D is diagonal P is invertible. If B is the basis for $\mathbb{R}^{n}$ formed from the
columns of P, called $\{\textbf{b}_{1},\textbf{b}_{2},\ldots,\textbf{b}_{n}\}$ are 
eigenvectors of A. D is the B-matrix for the transformation A $\to$ A\textbf{x}
\end{itemize}
\end{subsection}
\begin{subsection}*{Topics}
\begin{itemize}
\item{Difference Equations}
\item{Diagonalization}
\item{Basis Coordinates For Abstract Linear Transformation}

\end{itemize}
\end{subsection}
\begin{subsection}{Difference Equations}
A recursive relation of the form $\textbf{x}_{k+1}=A\textbf{x}_{k}$ for some matrix
A and $\textbf{x}_{1},\textbf{x}_{2},\textbf{x}_{3}$.\\
Assume we start with initial condition $\textbf{x}_{1}=A\textbf{x}_{0},~\textbf{x}_{2}
A\textbf{x}_{1}=AA\textbf{x}_{0},~ \textbf{x}_{k}=A\textbf{x}_{k-1}=
A^{k}\textbf{x}_{0}$
\end{subsection}
\begin{subsection}{Diagonalization}
\begin{equation} \label{Diagonal Exponents}
D^{k}=\begin{bmatrix} a_{11}^{k} &0 &0 \\0& a_{22}^{k} & 0\\ 0& 0& a_{33}^{k} 
\end{bmatrix} \to A=P*D*P^{-1} \mbox{ (Similarity)}
\end{equation}
\textbf{Diagonalizable Definition}:A square matrix A is said to be diagonalizable if
A is similar to a diagonal matrix. AKA $A=PDP^{-1}$ for some invertible P and some 
diagonal D.\\
\begin{equation}
P=[\textbf{v}_{1}, \textbf{v}_{2}, \textbf{v}_{n}] \to A\textbf{v}_{i}=\lambda_{i}
\textbf{v}_{i} \to D= \begin{bmatrix} x_1 & \ldots & \ldots\\ \ldots & x_2 & \ldots
\\ \ldots & \ldots &x_3\end{bmatrix} 
\end{equation}
\textbf{Finding $A=PDP^{-1}$:}
\begin{enumerate}
\item{Find the eigenvalues}
\item{Find n linearly independent eigenvectors of A}
\item{Construct P from the vectors in step 2}
\item{Construct D from the eigenvalues corresponding to P's columns}
\item{Invert P, run the function $A=PDP^{-1}$}
\item{If this was done correctly, AP=PD}
\end{enumerate}
\end{subsection}
\begin{subsection}{Basis Coordinates for Abstract Linear Transformation}
We can use basis coordinates to relate abstract transformations to matrix 
multiplication on vectors in real spaces.\\
V is an n-dimensional vector space and W is an m-dimensional vector space. 
\begin{itemize}
\item{B is a basis for V, the B-coordinate vector $x\in V \to[x]_{b}\in 
\mathbb{R}^{n}$}
\item{C is a basis for W, the C-coordinate vector $u\in w \to[u]_{C}\in 
\mathbb{R}^{m}$}
\item{$\{\textbf{b}_{1},\dots,\textbf{b}_{n}\}$ is the basis B for V}
\item{if $\textbf{x}=r_1\textbf{b}_{1}+\dots+r_{n}\textbf{b}_{n}$ then 
$[x]_{B}=\begin{bmatrix} r_1\\ r_2\\ \dots \\r_{n} \end{bmatrix} $}
\end{itemize}
T:V $\to$ W is a linear transformation.
\begin{equation}
T(\textbf{x})=T(r_1\textbf{b}_{1}+\dots+r_{n}\textbf{b}_{n})=
T(\textbf{b}_{1})+\dots+T(\textbf{b}_{n}) \in W
\end{equation}
Since the C coordinate mapping from W to to $\mathbb{R}^{m}$ is linear:
\begin{equation}
[T(\textbf{x})]_c=[r_1T(\textbf{b}_{1})+\dots+r_{n}T(\textbf{b}_{n})]_C=
r_1[T(\textbf{b}_{1})]_{C}+\dots+r_{n}[T(\textbf{b}_{n})]_{C} \in \mathbb{R}^{n}
\end{equation}
\begin{equation}
[T(x)]_{C}=M[x]_{B} \to M=[[T(\textbf{b}_1)]_{C},\dots,[T(\textbf{b}_{n})]_{C}
\end{equation}
M is the matrix for T relative to the bases B and C. Thus, you can find matrix M 
relative to B and C\\
T: V $\to$ V LT the basis C is the same as the basis B so M is the matrix relative to
B T[B] $\to [T(\textbf{x})]_{B}=[T]_{B}[x]_{B}[T]_{B}$
\\
Linear transformations $\in \mathbb{R}^{n}$: a linear transformation T represented 
by matrix multiplication where $\textbf{x} \to A\textbf{x}$\\
If A is diagonalizable, then there is a basis B for $\mathbb{R}^{n}$ consisting of 
eigenvectors of A.\\
It turns out that the B matrix for T is diagonal.
\end{subsection}
\end{section}
\newpage
\begin{section}{Inner Products and Orthogonality}
\begin{subsection}*{Theorems}
\textbf{Theorem 6.1}: \textbf{u}, \textbf{v}, \textbf{w} are vectors in 
$\mathbb{R}^{n}$, $C \in \mathbb{R}$
\begin{enumerate}
\item{$\textbf{u}*\textbf{v}=\textbf{v}*\textbf{u}$}
\item{$(\textbf{u}+\textbf{v})*\textbf{w}=\textbf{u}\textbf{w}+\textbf{v}+
	\textbf{w}$}
\item{$(c\textbf{u})*\textbf{v}=c(\textbf{u}\textbf{v}$}
\item{$\textbf{u}-\textbf{u} \geq \textbf{0} \and \textbf{u}*\textbf{u}=\textbf{0}
iff \textbf{u}=\textbf{0}$}
\end{enumerate} 
\noindent \textbf{Theorem 6.2 - Pythagorean Theorem}: Two vectors \textbf{u} and 
\textbf{v} are orthogonal iff 
\begin{equation} \label{}
	||\textbf{u} +\textbf{v}||^{2}=|\textbf{u}|^{2}+|\textbf{v}|^{2}
\end{equation}
\\ If \textbf{u}, \textbf{v} are not orthogonal, then $||\textbf{u} +
\textbf{v}||^{2} \geq |\textbf{u}|^{2}+|\textbf{v}|^{2}$
\end{subsection}
\noindent \textbf{Theorem 6.4}: If $S=\{\textbf{u}_{1},\ldots,\textbf{u}_{p}\}$ 
is an orthogonal set of nonzero vectors in $\mathbb{R}^{n}$, then S is linearly
independent and is a basis for the subspace spanned by S. $\textbf{0}=
c_1\textbf{u}_{1}+\ldots+c_{p}\textbf{u}_{p} \to \textbf{0}\cdot \textbf{u}_{1}=
textbf{u}_{1}(c_1\textbf{u}_{1})+\ldots+textbf{u}_{1}(c_{p}\textbf{u}_{p}) \to
c||u_1||^{2}=0$ \\
\noindent \textbf{Theorem 6.5}: let $\{\textbf{u}_{1},\ldots,\textbf{u}_{p}\}$ be
an orthogonal basis for a subspace W of $\mathbb{R}^{n}$. For each \textbf{y} in 
W, the weights in the linear combination $\textbf{y}=c_1\textbf{u}_{1}+\ldots+
c_{p}\textbf{u}_{p}$ are given by $\textbf{c}_{1}=\frac{\textbf{y}\cdot
\textbf{u}_{1}}{\textbf{u}_{1}\cdot\textbf{u}_{1}}$
\begin{equation}
	[\textbf{y}]_{B}=\begin{bmatrix} \frac{\textbf{y}\cdot
\textbf{u}_{1}}{\textbf{u}_{1}\cdot\textbf{u}_{1}} \\ \dots \\
\frac{\textbf{y}\cdot \textbf{u}_{p}}{\textbf{u}_{p}\cdot\textbf{u}_{p}} 
\end{bmatrix} 
\end{equation}
\\ \noindent \textbf{Theorem 6.8 - The Orthogonal Decomposition Theorem}:
Let W be a subspace of $\mathbb{R}^{n}$. Then every y in $\mathbb{R}^{n}$ can be
written uniquely in the form $\textbf{y}=\hat{y} + z$. Where $\hat{y} \in W$ and 
$z \in W^{\perp}$, and if $\{\textbf{u}_{1},\ldots,\textbf{u}_{p}\}$ is an 
orthogonal basis of W. 
\begin{equation}
\hat{y}=\frac{\textbf{y}\cdot\textbf{u}_{1}}{\textbf{u}_{1}\cdot\textbf{u}_{1}}+
\dots +
\frac{\textbf{y}\cdot\textbf{u}_{p}}{\textbf{u}_{p}\cdot\textbf{u}_{p}}
\end{equation}
\\ \noindent \textbf{Theorem 6.9 - The Best Approximation}:W is a subspace of
$\mathbb{R}^{n}$, let \textbf{y} be any vector $\in \mathbb{R}^{n}$. $\hat{y}$ is
the orthogonal projection of \textbf{y} into W. Then $\hat{y}$ is the closest
point in  W to \textbf{y}. $||\textbf{y}-\hat{y}||\leq||\textbf{y}-\textbf{v}||$
for all \textbf{v} in W distinct from $\hat{y}$
\begin{subsection}*{Topics}
\begin{itemize}
\item{Inner Product}
\end{itemize}
\end{subsection}
\begin{subsection}{Inner Product}
let \textbf{u} and \textbf{v} be vectors in $\mathbb{R}^{n}$. Consider them as
n*1 matrices. The Transpose $\textbf{u}^{t}$ is a 1*n matrix. The matrix product,
$\textbf{u}^{t}\textbf{v}$ is a 1*1 matrix, scalar - the inner product. \\
\begin{equation}
\textbf{u}=\begin{bmatrix} u_1 \\ u_2 \\ \dots \\ u_{n} \end{bmatrix} 
\textbf{v}= \begin{bmatrix} v_1 \\ v_2 \\ \dots \\ v_{n} \end{bmatrix}
\textbf{u}^{t}*\textbf{v}=u_1v_1+u_2v_2+\dots +u_{n}v_{n}
\end{equation}
\\ \textbf{Definition - Length}: The length of n or m of \textbf{v} is the 
non-negative scalar ||\textbf{v}||. ||\textbf{v}||
$= \sqrt{\textbf{v}^{T}\textbf{v}}$
\\ \textbf{Definition - Unit Vector}: A vector whose length is 1 is called a unit 
vector ||\textbf{u}||=1. $\textbf{u}=\frac{1}{||\textbf{v}||}\textbf{v}$ is the
normalized vector. \\
\textbf{Definition - Distance}: For \textbf{u}, \textbf{v} $\in \mathbb{R}^{n}$,
the distance between \textbf{u} \and \textbf{v}, written as dist(\textbf{u},
\textbf{v}) is the length of \textbf{u}-\textbf{v}.
\begin{equation}
	\mbox{dist}(\textbf{u},\textbf{v}) = ||\textbf{u}-\textbf{v}||
\end{equation} \\
\textbf{Definition - Orthogonal}: \textbf{u},\textbf{v} $\in \mathbb{R}^{n}$ are
orthogonal (to each other) if $\textbf{u}\cdot \textbf{v} = \textbf{0}$.
\\ \textbf{Definition - Orthogonal set:} A set of vectors $\{ \textbf{u}_{1},\dots
, \textbf{u}_{n}\} \in \mathbb{R}^{n}$ is an orthogonal set if each pair of 
distinct vectors from the set is orthogonal. $\textbf{u}_{i}\cdot\textbf{u}_{j}=
\textbf{0}$ \\
Subspace W is also a subspace W $\in \mathbb{R}^{n}$ if a vector z is orthogonal
to every vector in W, Z is orthogonal to W. The set of all vectors Z orthogonal to
W is the orthogonal complement of W, denoted $W^{\perp}$ \\
\textbf{Definition - Orthogonal Basis}: An orthogonal basis for a subspace W of
$\mathbb{R}^{n}$ is a basis for W that is also an orthogonal set. 
\\ An given vector \textbf{u} has a vector y-hat within its span
and a vector \textbf{z}, orthogonal to it. From this,
\begin{equation} 
\textbf{y}=\hat{y}+\textbf{z} \to 
\hat{y}=\alpha\textbf{u} \to
0=z\cdot \textbf{u}=(y-\alpha\textbf{u})\cdot\textbf{u} \to \alpha=\frac
{\textbf{y}\cdot\textbf{u}}{\textbf{u}\cdot\textbf{u}} \to \hat{y}=\frac
{\textbf{y}\cdot\textbf{u}}{\textbf{u}\cdot\textbf{u}}=\mbox{proj}_{u}\textbf{y}
\end{equation}
\\ This is called an orthogonal decomposition.
\end{subsection}
\end{section}
\newpage
\begin{section}{Gram-Schmidt Process and Least Squares}
\begin{subsection}*{Theorems}
\textbf{Theorem 6.11 - Gram-Schmidt Process}: Given a basis $\{\textbf{x}_{1}
,\textbf{x}_{2},\dots,\textbf{x}_{3}\}$for a nonzero subspace W of $\mathbb{R}^{n}$ \\
\begin{equation}
v_1=x_1 \to 
v_2=x_2-\frac{x_2\cdot v_1}{v_1\cdot v_1}*v_1
\end{equation}
\begin{equation}
v_{p}=x_{p}-\frac{x_{p}\cdot v_1}{v_1\cdot v_1}*v_1 - 
\frac{x_{p} \cdot v_2}{v_2 \cdot v_2}*v_2
-\frac{x_{p} \cdot v_{p-1}}{v_{p-1}\cdot v_{p-1}*v_{p-1}}*v_{p-1}
\end{equation}

Then $\{\textbf{v}_{1},\dots,\textbf{v}_{P}\}$ is an orthogonal basis for W and 
\noindent \textbf{Theorem 6.13}: The set of least-squares solutions of $A\textbf{x}
=\textbf{b}$ coincides with the non-empty set of solution to the normal equations
\begin{equation} \label{Least Square}
A^{T}A\textbf{x}=A^{T}\textbf{b}
\end{equation}
\\ \noindent \textbf{Theorem 6.14}: Let A be an m*n matrix. The following
statements are logically equivalent:
\begin{itemize}
\item{The equation $A\textbf{x}=\textbf{b}$ has a unique least squares solution
for each $\textbf{b}\in\mathbb{R}^{m}$}
\item{The columns of A are linearly independent}
\item{The matrix $A^{T}A$ is invertible}
\end{itemize}
When these statements are true, the least squares solution $\hat{x}$ is given by
\begin{equation}
	\hat{x}=(A^{T}A)^{-1}A^{T}\textbf{b}
\end{equation}
\end{subsection}
\begin{subsection}*{Topics}

\end{subsection}
\begin{subsection}{Gram-Schmidt}
With any $\{\textbf{v}_{1},\dots,\textbf{v}_{n}\}$ that is an LI basis for 
Vector Space W, you can make $\{\textbf{u}_{1},\dots,\textbf{e}_{n}\}$ \\
\textbf{Gram-Schmidt Process}
\begin{enumerate}
\item{Let W=Span$\{\textbf{x}_{1}, \textbf{x}_{2}, \textbf{x}_{3}\}$}
\item{$\textbf{v}_{1}=\textbf{x}_{1},~W_1=Span\{\textbf{v}_{1}
	,\textbf{v}_{2},\textbf{v}_{3}\}$}
\item{$\textbf{v}_{2}=\textbf{x}_{2}-proj_{w1}\textbf{x}_{2}=\textbf{x
}_{2}-\frac{\textbf{x}_{2}\cdot\textbf{v}_{1}}{\textbf{v}_{1}\cdot
\textbf{v}_{1}}*\textbf{v}_{1}$}
\item{$\textbf{v}_{3}=\textbf{x}_{3}-proj_{w2}\textbf{x}_{3}=\textbf{x
}_{3}-\frac{\textbf{x}_{3}\cdot\textbf{v}_{1}}{\textbf{v}_{1}\cdot
\textbf{v}_{1}}*\textbf{v}_{1}+\frac{\textbf{x}_{3}\cdot\textbf{v}_{2}}
{\textbf{v}_{2}\cdot\textbf{v}_{2}}*\textbf{v}_{2}$}
\end{enumerate}
\end{subsection}
\begin{subsection}{Least Squares}
In applications, A\textbf{x}=\textbf{b} will most often be inconsistent (Sad). \\
\textbf{Definition - Least Squares}: If A is an m*n matrix and \textbf{b}
$\mathbb{R}^{n}$, a least squares solution of A\textbf{x}=\textbf{b}, is an 
$\hat{x} \in \mathbb{R}^{n}$ such that \\
$||\textbf{b}-A\hat{x}|| \leq ||b-A\textbf{x}||$ for all x in $\mathbb{R}^{n}$ \\
No matter what \textbf{x} we pick, A\textbf{x} will be in the column space of A.
Find the \textbf{x} that makes A\textbf{x} the Closest space in ColA to \textbf{b}
\\ Apply the best approximation theorem:\\
$\hat{b}=proj_{colA}\textbf{b}$ - Because $\hat{b}$ is in the column space of A,
$A\textbf{x}=\hat{b}$ is consistent. Thus, there exists an $\hat{x}$ such that 
$A\hat{x}=\hat{b}$ From this, $\hat{x}$ is a list of weights that builds $\hat{b}$
out of the columns of A. \\
Suppose $\hat{x}$ satisfies $A\textbf{x}=\hat{b}$. By the orthogonal decomposition
theorem: $\textbf{b}-\hat{b}$ is orthogonal to ColA, meaning $\textbf{b}-\hat{b}$ 
is orthogonal to column of A. \\
$a_{j}*(\textbf{b}-\hat{b})=0 \to a_{j}^{T}(\textbf{b}-\hat{b}) \to A^{T}
(\textbf{b}-\hat{b}) = A^{T}(\textbf{b}-A\hat{x}=0$\\
$A^{T}A\hat{x}=A^{T}\textbf{b}$
\end{subsection}
\end{section}
\end{document}
