\documentclass[a4paper, 12pt]{article}

\usepackage[utf8]{inputenc}
\usepackage[T1]{fontenc}
\usepackage{textcomp}
\usepackage[dutch]{babel}
\usepackage{amsmath, amssymb}
\usepackage[left=20mm, right=20mm, top=10mm, bottom=20mm]{geometry}

\pdfsuppresswarningpagegroup=1

\begin{document}
\begin{section}{What is Linear Algebra}
	\begin{subsection}*{Topics}
\begin{enumerate}
		\item{What Linear Algebra is}
		\item{Elementary Row Operations (ERO's)}
		\item{Echelon form}
\end{enumerate}
	\end{subsection}
	\begin{subsection}{What is Linear Algebra?}
	\textbf{The Basic Definition:} Linear Algebra
	is the study of solving linear equations.\\
	These generally come in the form of
	$a_1*x_1+a_2*x_2+...+a_n*x_n=b$\\
	We are looking for intersections. This can be 
	done through substitution, general algebraic methods
	(ERO's) and matrix rotation. \\
\end{subsection}
\begin{subsection}{Elementary Row Operations (Three Main Types)}
\begin{enumerate}
	\item{\textbf{Replacement}: Replace one row by the 
		sum of itself & a scalar multiple of another.}
	\item{\textbf{Interchange}: Switch two rows.}
	\item{\textbf{Scaling}: Multiply all entries in a row by
		a non-zero constant}
\end{enumerate}
\end{subsection}
\begin{subsection}{Echelon Forms}
	\textbf{Echelon Form}: All non-zero rows are above rows
	of all zeroes.
	\begin{itemize}
	\item{All leading entries of a row is in a column to 
		the right of the row above it}
	\item{All entries are in a column below a leading entry
		are zeroes}
	\end{itemize}
	\\ \textbf{Reduced Echelon Form}: A unique solution!
	\begin{itemize}
\item{The leading entry in each non-zero row is a 1}
\item{Each leading 1 is the only non-zero entry in the column}
	\end{itemize}
\end{subsection}

\begin{subsection}{Miscellaneous Notes from Day 1}
	\begin{itemize}
		\item{\textbf{An Intersection of Lines}:
			unique solution}
		\item{\textbf{Equivalent Lines}:
			infinite solutions}
		\item{\textbf{Parallel lines}: no solution}
		\item{\textbf{Consistent}: $\geq 1$ solutions}
		\item{\textbf{Inconsistent}: 0 Solutions}
	\end{itemize}
\end{subsection}

\end{section}	

\begin{section}{Vector and Matrix Equations}

\begin{subsection}*{Theorems}
\textbf{Theorem 3}: If A is an m*n matrix with columns
$\textbf{a}_1, \textbf{a}_2,\dots,\textbf{a}_n$ & if 
$b \in \mathbb{R}^n$, then matrix equation $A\textbf{x}=\textbf{b}$
has the same solution as the vector equation
$x_1\textbf{a}_1+x_2\textbf{a}_2+\dots+x_n\textbf{a}_n=\textbf{b}$
and the system with the augmented matrix 
$\begin{bmatrix} 
\textbf{a}_1 & \textbf{a}_2 & \textbf{a}_n & \textbf{b} \end{bmatrix}$ 
\\ \\
\textbf{Theorem 4}: For an m*n matrix A the following statements
are logically equivalent
\begin{enumerate}
	\item{For all $\textbf{b}\in\mathbb{R}^m$ 
	there exists a solution for $A\textbf{x}=\textbf{b}$}
	\item{Each $\textbf{b}\in\mathbb{R}^m$ is a linear combination
		of A columns}
	\item{The columns of A span $\mathbb{R}^m$}
	\item{A has a pivot in each row}
\end{enumerate}
\end{subsection}
\begin{subsection}*{Topics}
\begin{enumerate}
	\item{Defining vectors and the span of a collection of vectors}
	\item{Writing linear equations as a matrix * vector}
	\item{Recognizing $M*\textbf{v}$ as a linear combination
		of columns of a matrix}
	\item{Algebraic Properties of Vectors}
\end{enumerate}
\end{subsection}

\begin{subsection}{Span and vectors}
	\textbf{Vectors}: A matrix with one column and m dimensions
	where m is #of rows.
\begin{itemize}
	\item{$\textbf{u}=\langle1,2\rangle$ 
	defines $\mathbb{R}^2$}
	\item{$\textbf{v}=\langle1,2,-5\rangle$ defines $\mathbb{R}^3$
\end{itemize}
\\ \textbf{Linear Combinations of Vectors}: a set where 
$\textbf{y}=c_1\textbf{v}_1+c_2\textbf{v}_2+\dots+c_n\textbf{v}_n$
where $(c_1,c_2,\dots,c_p)$ are scalars (weights)\\

\\ \noindent\textbf{Span}: The set of all linear combinations Span
    	$\lbrace \textbf{v}_1 , \dots , \textbf{v}_p \rbrace$ where 
	$\textbf{b}=c_1\textbf{v}_1+c_2\textbf{v}_2+\dots+
	c_p\textbf{v}_p \in \mathbb{R}$
\begin{itemize}
	\item{e.g span of $\textbf{v} \in \mathbb{R}^2$ is a line}
	\item{e.g span of $\textbf{v}+\textbf{u} 
	\in \mathbb{R}^3$ is a plane}
\end{itemize}
By the definition of 
$A\textbf{x}=x_1\textbf{a}_1+\dots+\x_n\textbf{a}_n$ through Theorem 3,
$A\textbf{x}=\textbf{b}$ has a solution iff \textbf{b} is the span of A
columns.
\end{subsection}
\begin{subsection}{Algebraic Properties of Vectors}
	For all \textbf{u},\textbf{v},\textbf{w} $\in \mathbb{R}^n$
	where $c_1 , c_2$ are scalars, these are true:
\begin{enumerate}
	\item{$\textbf{v}+\textbf{u}=\textbf{u}+\textbf{v}
	\Rightarrow (\textbf{u}+\textbf{v})+\textbf{w}=
	\textbf{u}+(\textbf{v}+\textbf{w})$
\item{$\textbf{u}+\textbf{0}=\textbf{u}$}
\item{$c(\textbf{u}+\textbf{v})=c\textbf{u}+c\textbf{v}$}
\item{$(c+d)\textbf{u}=c\textbf{u}+d\textbf{u}$}
\item{There is no $\textbf{u}*\textbf{v}$}
\end{enumerate}
\end{subsection}
\end{section}

\newpage
\begin{section}{Homogenous Systems and Linear Independence}

\begin{subsection}*{Theorems}
\textbf{Theorem 6}: Suppose the equation $A\textbf{x}=\textbf{b}$ is
consistent for some \textbf{b}, then the solution set of all the 
vectors of the form is $\textbf{w}=\textbf{p}+\textbf{v}_n$, where
$\textbf{v}_n$ is any solution of the homegenous 
system $A\textbf{x}=\textbf{0}$
\\
\\ \textbf{Theorem 7}: A set of 2+ vectors $S:\lbrace\textbf{v}_1,\dots
,\textbf{v}_n\rbrace$ is linearly dependent (LD) iff $\geq 1$ vector 
in S is a linear combination of the others.\\
\\
\textbf{Theorem 8}:if a set $\in\mathbb{R}^n$ contains more vectors 
than dimensions, then the set is LD.\\
(This is because there will be a free variable and thus not a single
unique solution)
\\
\\ \textbf{Theorem 9}: If a set $S:\lbrace\textbf{v}_1,/dots
,\textbf{v}_n\rbrace \in\mathbb{R}^n$ contains a \textbf{0} vector, the
set is LD
\end{subsection}
\begin{subsection}*{Topics}
	\begin{enumerate}
	\item{Homogenous Systems}
	\item{Parametric Form}
	\item{Linear Independence and Dependence}
	\end{enumerate}
\end{subsection}
\begin{subsection}{Homogenous Systems}
When $A\textbf{x}=\textbf{0}$, the trivial solution is
$\textbf{x}=\textbf{0}$. The non-trivial form of this is when
\textbf{x} is equal to something other then \textbf{0}.\\
The solution set to a homogenous system is the span of some set of 
vectors $S:\lbrace\textbf{v}_1,\textbf{v}_n\rbrace$
\begin{itemize}
\item{The system $A\textbf{x}=\textbf{0}$
	will have a non-trivial solution if there is a free
	variable within the coefficient matrix}
\end{itemize}

\end{subsection}

\begin{subsection}{Parametric Form}
\begin{enumerate}
	\item{Row reduce augmented matrix to reudced echelon form}
	\item{Express base cariables in terms of free variables in 
		equation system}
	\item{Write solution vector \textbf{x}whose entries
		depend on free variables}
	\item{decompose \textbf{x} into linear combination w/ free 
		variable as the parameter for one of the vectors}
\end{enumerate}
\begin{equation*}{Example:}
\begin{bmatrix} 
	1 & 0 & 2 & 2 \\
	0 & 1 & 3 & 4 \\
	0 & 0 & 0 & 0
\end{bmatrix} 
\Rightarrow
x_3\begin{bmatrix} -2 \\ -3 \\ 1 \end{bmatrix} +
\begin{bmatrix} 2 \\ 4 \\ 0 \end{bmatrix} 
\end{equation*}
\end{subsection}

\begin{subsection}{Linear Independence and Dependence}
\textbf{LI}:
A set of vectors $\in \mathbb{R}^n$ is Linearly Independent
(LI) if the vector equation
$x_1\textbf{v}_1+\dots+x_n\textbf{v}_n=\textbf{0}$ can only
be solved with the trivial solution $\textbf{x}=\textbf{0}$
\\

\noindent \textbf{LD}:
A set of vectors $\in \mathbb{R}^n$ is Linearly Dependent(LD) if 
there exist weights $(c_1,c_2,\dots,c_n)$, not all zero where
$\lbrace c_1\textbf{v}_1+c_2\textbf{v}_2+\dots+c_n\textbf{v}_n\rbrace=0$
\begin{itemize}
\item{$m>n=#col>#row=#vectors>#dimenstions$}
\item{If a vector is a scalar multiple of another in the set, LD}
\end{itemize}
\end{subsection}
\end{section}

\begin{section}{Linear Transformations and the Standard Matrix}
\begin{subsection}*{Theorems}
\textbf{Theorem 5}: If a matrix is m*n, then the transform 
$\textbf{x} \Rightarrow A\textbf{x}$ has these properties:
\begin{enumerate}
\item{$A(\textbf{u}+\textbf{v})=A\textbf{u}+A\textbf{v}$
Vector addition is preserved}
\item{$A(c\textbf{u})=cA\textbf{u}$} Scalar multiplication
	is preserved.
\end{enumerate}
\\
\textbf{Theorem 10}: If T is a linear transformation
$\mathbb{R}^n \Rightarrow \mathbb{R}^m$ then there exists a matrix
A with dimensions m*n such that $T(\textbf{x})=A\textbf{x}$ for
all $\textbf{x}\in\mathbb{R}$ 
\\
$A= \begin{bmatrix} T(\textbf{e}_1) & \dots & T(\textbf{e}_n) 
\end{bmatrix}$ whose jth column is $T(\textbf{E}_j)$ where 
$\textbf{e}_j$ is the jth column of the identity matrix in 
$\mathbb{R}^n$
\\
\\ \textbf{Theorem 11}: Let T:$\mathbb{R}^n\Rightarrow\mathbb{R}^m$
be a linear transformation. Then T is one-to-one iff the equation $T(\textbf{x})=0$ has only
a trivial solution (LI)
\\
\\ \textbf{Theorem 12}:
Let T:$\mathbb{R}^n\Rightarrow\mathbb{R}^m$
be a linear transformation. A is the standard matrix of T
\begin{itemize}
\item{T maps $\mathbb{R}^n\Rightarrow\mathbb{R}^m$ iff columns of
	A span $\mathbb{R}^m$}
\item{T is one-to-one iff columns's of A are LI}
\end{itemize}
\end{subsection}
	
\begin{subsection}*{Topics}
\begin{enumerate}
\item{Linear Transformations}
\item{Transformation Groups}
\item{One-to-one and Onto}
\end{enumerate}
\end{subsection}

\begin{subsection}{Linear Transformations}
Linear transformation allows for the mapping from one set
of vectors in a real space, to another set, in another
real space.\\
\textbf{Matrix Mapping}: The process of going from one set in one
space to another set in another space. A T transform is a function
of matrix mapping.\\
From this, a T from $\mathbb{R}^n \Rightarrow \mathbb{R}^m$ is a 
rule that assigns a vector $\textbf{x}\in\mathbb{R}^n$ to a vector
in $\mathbb{R}^m$.\\
\\ \textbf{Transforms $T:\mathbb{R}^n \Rightarrow \mathbb{R}^m$}
\begin{enumerate}
\item{Set $\mathbb{R}^n$ is in the \textbf{domain} of T}
\item{Set $\mathbb{R}^m$ is in the \textbf{codomain} of T}
\item{For an $\textbf{x}\in\mathbb{R}^n$, the vector 
$T\textbf{x}\in\mathbb{R}^m$ is called the \textbf{image} of 
\textbf{x} under the action T}
\item{The set of all images is called the \textbf{range} of T}
\end{enumerate}
T(\textbf{x}) formula: Linear Transformation
$\mathbb{R}^n \Rightarrow \mathbb{R}^m$ is actually a matrix
transformation $A\textbf{x}\Rightarrow\textbf{x}$. The key is to 
look at what T does to the columns of the identity matrix.
\end{subsection}

\begin{subsection}{Identity Matrices and Transformation Groups}
\textbf{Identity Matrix}: 
$\begin{bmatrix} 1 & 0 \\ 0 & 1 \end{bmatrix} 
\begin{bmatrix} \textbf{x}_1 \\ \textbf{x}_2 \end{bmatrix} 
= \begin{bmatrix} \textbf{x}_1 \\ \textbf{x}_2 \end{bmatrix} $
\\
\\ \textbf{Group 1}: Contraction and Expansion 
$\begin{bmatrix} K & 0 \\ 0 & 1 \end{bmatrix} & 
\begin{bmatrix} 1 & 0 \\ 0 & K \end{bmatrix}$
\\
\textbf{Group 2}: Reflection 
$\begin{bmatrix} 0 & 1 \\ 1 & 0 \end{bmatrix}$
\\
\textbf{Group 3}: Shearing
$\begin{bmatrix} 1 & K \\ 0 & 1 \end{bmatrix}
\begin{bmatrix} 1 & 0 \\ K & 1\end{bmatrix}  $
\\
\textbf{Group 4}: Rotations
$\begin{bmatrix} cos\Theta & -sin\Theta \\
sin\Theta & cos\Theta \end{bmatrix} $
\\
\textbf{Group 5}: Axis Projection
$\begin{bmatrix} 1 & 0 \\ 0 & 0 \end{bmatrix} 
\begin{bmatrix} 0 & 0 \\ 0 & 1 \end{bmatrix} $
\end{subsection}

\begin{subsection}{One-to-One and Onto}
\Large{\textbf{Onto}}: A mapping T: 
$\mathbb{R}^n\Rightarrow\mathbb{R}^m$ is \textbf{onto} if each
$\textbf{b}\in\mathbb{R}^m$ is the image of at least one
$\textbf{x}\in\mathbb{R}^m$ \\
\\ \noindent\Large{\textbf{One-to-one}}: A mapping 
T: $\mathbb{R}^n\Rightarrow\mathbb{R}^m$ is \textbf{one-to-one}
if each $\textbf{b}\in\mathbb{R}^m$ is the image of at most one
$\textbf{x}\in\mathbb{R}^m$
\end{subsection}
\end{section}

\newpage
\begin{section}{Matrix Multiplication and Invertability}
\begin{subsection}*{Theorems}
\textbf{Theorem 1}:
\begin{itemize}
\item{$(A^T)^T=A$}
\item{$(A+B)^T=A^T+B^T$}
\item{$(rA)^T=rA^T$}
\item{$(AB)^T=A^TB^T$}
\end{itemize}
\\
\textbf{Theorem 5}: If A is an n*n matrix & is invertible, then for every $\textbf{b}\in\mathbb{R}^n$ the equation
$A\textbf{x}=\textbf{b}$ has a unique solution: 
$A^{-1}\textbf{b}=\textbf{x}$
\\
\noindent\textbf{Theorem 6}: If A is invertible, assume:
\begin{itemize}
\item{$(A^{-1}^{-1}=A$}
\item{$(AB)^{-1}=A^{-1}B^{-1}$}
\item{$(A^{T})^{-1}=(A^{-1})^{T}$}
\end{itemize}
\\
\noindent\textbf{Theorem 7}: An n*n matrix $A\in\mathbb{R}^{n}$ is 
invertible iff A is row equivalent to $I_{n}$ \\
Any sequence of ERO's that transform $ A\Rightarrow I_{n}$ also 
transforms $I_{n} \Rightarrow A^{-1}$

\\
\end{subsection} 

\begin{subsection}*{Topics}
\begin{enumerate}
\item{Matrix Properties and Definition}
\item{Row Column Computation}
\item{Invertibility}
\end{enumerate} \end{subsection}

\begin{subsection}{Matrix Properties and Definitions}
A,B,C are Matrices where m=n and r&s are scalars
\begin{enumerate}
\item{$A+B=B+A$}
\item{$A+0=A$}
\item{$r(A+B)=rA+rB$}
\item{$rs(A)=rA+sA$}
\end{enumerate}
\\
A diagonal Matrix is a matrix where all non-diagonal entries are 0\\
A zero matrix is a matrix where all entries are 0\\
A+B adds each corresponding entry $(a+b)_{ij}=a_{ij}+b_{ij}$\\
Scalar multiplication is preserved \\
Matrix multiplication isn't commutative\\
\begin{itemize}
\item{$AB=AC\neq B=C$}
\item{$(AB)C=A(BC)$}
\item{$AB=0$ doesn't mean $A=0 or B=0$}
\item{$AB\neq BA$}
\end{itemize}
A is m*n, B & C have operable dimensions
\begin{itemize}
\item{$A(BC)=(AB)C$}
\item{$A(B+C)=AB+AC$ Left Distributive}
\item{$(B+C)A=BA+CA$ Right Distributive}
\item{$r(AB)=rA(B)=A)rB)$ Scalars}
\item{$I_{m}A=A=AI_{n}$}
\end{itemize}
\end{subsection}

\begin{subsection}{Row Column Computation}
$B\textbf{x}$ transforms $\textbf{x} \Rightarrow B\textbf{x}$\\
Matrix A turn $B\textbf{x}\Rightarrow A(B\textbf{x})$\\
For this to work, A must be m*n and B must be n*p. The outcome matrix
is n*p. Acol must equal Brow.
\\
\textbf{Row column rule for computing AB}: If AB is defined Acol=Brow
then the entry at row i and column j of AB is
\begin{equation}
	(AB)_{ij}=a_{i_1}b_{1j}+a_{i_2}b_{21}+\dots+a_{in}b_{nj}
\end{equation}
\end{subsection}

\begin{subsection}{Invertibility}
\begin{equation}
	A^{-1}A=AA^{-1}=I
\end{equation}
If $A^{-1}$ exists for A, A is invertible.\\
Non-invertible $\Leftrightarrow$ singular\\
$A^{-1}$ is unique\\
\textbf{The Elementary Matrix}: Obtained bt performing a single ERO on an
identity matrix.\\
An ERO becomes an EA by performing operations on I.\\
The inverse of E is the elementary matrix of the same type that transforms
E back into I.\\ 
Apply ERO's both to A and I at the same time in one augmented matrix gives
$A^{-1}$ once A is in RREF.
\end{subsection}

\end{section}

\newpage
\begin{section}{Determinants and the Invertible Matrix Theorem}
\begin{subsection}*{Theorems}
\textbf{Theorem 2.8}: The Invertible Matrix Theorem (IMT) -- Let A be a
square n*n matrix.
\begin{enumerate}
\item{If any statements below aretrue, all are true. 
If any statements below are false all are false}
\begin{enumerate}
\item{A is an invertible matrix}
\item{A is row equivalent to the n*n identity matrix}
\item{A has n pivot positions}
\item{the equation $A\textbf{x}=\textbf{0}$ has only the trivial solution}
\item{The columns of A form a linearly independent set}
\item{The linear transformation $\textbf{x}\Rightarrow A\textbf{x}$ 
	is One-to-One}
\item{The equation $A\textbf{x}=\textbf{b}$ has at least one
	solution for each $\textbf{b}\in\mathbb{R}^n$}
\item{The columns of A span $\mathbb{R}^n$}
\item{The linear transformation $\textbf{x}\Rightarrow A\textbf{x}$
	maps $\mathbb{R}^n$ onto $\mathbb{R}^n$}
\item{There is an n*n matrix C such that CA=I}
\item{There is an n*n matrix D such that AD=I}
\item{$A^T$ is an invertible matrix}
\end{enumerate} \end{enumerate}
\textbf{Theorem 2.9}: Let T:$\mathbb{R}^n\Rightarrow\mathbb{R}^n$
be a linear transformation and let A be the standard matrix of T,
then T is invertible iff A is an invertible matrix.
\\
\\\textbf{Theorem 2.5}: let 
$A=\begin{bmatrix} a_{11} & a_{12} \\ a_{21} & a_{22} 
\end{bmatrix}$, if $ad-bd=0$ then A is invertible and 
\\$A^{-1}=\frac{1}{a_{11}a_{22}-a_{12}a_{21}}*
\begin{bmatrix} a_{22} & -a_{12} \\ -a_{21} & a_{11} \end{bmatrix}$\\
If $a_{11}a_{22}-a_{12}a_{21}=0$ then A is not invertible
detA$=a_{11}a_{22}-a_{12}a_{21}$
\\
\\ \textbf{Theorem 3.1}: The determinant of an n*n matrix A can be
computed by a cofactor expansion across any row or down any column.
\\ The cofactor expansion across the ith row:\\
$detA=a_{i1}c_{i1}+a_{i2}C_{12}+\dots+a_{in}C_{in}$\\
The Cofactor expansion down the jth column:\\
$detA=a_{1j}C_{1j}+a_{2j}C_{2j}+\dots+a_{nj}C_{nj}$
\\
\textbf{Theorem 3.2}: If A is a triangular matrix, then detA is the 
product of the entries on the main diagonal
\end{subsection}
\begin{subsection}{Topics}
\begin{enumerate}
\item{Invertability, Span, LI, Pivots, in the context of the
	Invertible Matrix Theorem (IMT)}
\item{Determinants of 3x3 matrices and NxN matrices}
\end{enumerate}
\end{subsection}

\begin{subsection}{Determinants}
\begin{equation}
	a_{11}\neq0\\
\begin{bmatrix} a_{11} & a_{12} & a_{13} \\ a_{21} & a_{22} & a_{23}
\\ a_{31} & a_{32} & a_{33} \end{bmatrix} \Rightarrow
\begin{bmatrix} a_{11} & a_{12} & a_{13} \\
0 & a_{11}a_{22}-a_{12}a_{21} & a_{11}a_{23}-a_{13}a_{21} \\
0 & 0 & a_{11}\Delta \end{bmatrix} \\ \\
\end{equation}
\begin{equation}
\Delta=a_{11}a_{22}a_{33}+a_{12}a_{23}a_{31}+a_{12}a_{21}a_{32}-
a_{11}a_{23}a_{31}-a_{12}a_{21}a_{33}-a_{13}a_{22}a_{31}
\end{equation}
\begin{equation}
A invertible \Leftarrow\Rightarrow \Delta\neq0 \Rightarrow detA=\Delta
\end{equation}
\begin{equation}
\Delta = a_{11}det\begin{bmatrix} a_{22} & a_{23}
\\ a_{32} & a_{33} \end{bmatrix} 
-a_{12}det\begin{bmatrix} a_{21} & a_{23} \\ a_{31} & a_{33} 
\end{bmatrix} 
+a_{13}det\begin{bmatrix} a_{21} & a_{22} \\
a_{31} & a_{32} \end{bmatrix} 
\end{equation}

\noindent\textbf{Determinant Definition}: For n=2, the determinant of an n*n
matrix A is \\ 
\begin{equation}
detA=a_{11}detA_{11}-A_{12}detA_{12}+\dots+(-1)^{1+n} a_{1n}detA_{1n}=
\sum_{j=1}^{n}(-1)^{1+j}a_{ij}detA_{ij}
\end{equation}
\\ \\
\noindent\textbf{Cofactor}: Define th(ij) - cofactor of A as the number 
$C_{ij}$
\begin{equation}
	C_{ij}=(-1)^{i+j}detA_{ij} \Rightarrow
	detA= a_{11}c_{11}+a_{12}c_{12}+\dots+a_{1n}c_{1n}
\end{equation}
The cofactor expansion across the 1st row of A
\begin{equation}
	detA=a_{11}C_{11}+a_{12}C_{12}+\dots+a_{1n}C_{1n}	
\end{equation}
The cofactor expansion across the 1st column of A
\begin{equation}
	detA=a_{11}C_{11}+a_{21}C_{21}+\dots+a_{n1}C_{n1}	
\end{equation}
\textbf{Triangular Matrix}: A matrix that has all zeros
either below or above the main diagonal. \\
Upper triangular: 0's below the diangonal\\
Lower Triangular: 0's above the diagonal
\end{subsection}

\begin{subsection}{IMT Patterns}
$a\Rightarrow j\Rightarrow d\Rightarrow c\Rightarrow b\Rightarrow 
a\\
a\Rightarrow k\Rightarrow g\Rightarrow a\\
g\Leftrightarrow h\Leftrightarrow i \\
d\Leftrightarrow e\Leftrightarrow f \\
a\Leftrightarrow l$
\\
See Theorem 8 for more
\end{subsection}

\end{section}
\end{document}
